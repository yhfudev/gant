%# -*- coding: utf-8 -*-
%!TEX encoding = UTF-8 Unicode
%!TEX TS-program = xelatex
% 以上设定默认使用 XeLaTex 编译,并指定 Unicode 编码,供 TeXShop 自动识别
% Author: Yunhui Fu <yhfudev@gmail.com>
% License: GPL v3.0
\documentclass[letter,12pt,onecolumn]{article}
%\documentclass[letter,12pt,onecolumn]{book}

\newcommand{\doctitle}{\cnt{Garmin Gadget}{高明小东东}{高明小東東}}
\newcommand{\docauthor}{Yunhui Fu}
\newcommand{\dockeywords}{}
\newcommand{\docsubject}{}

\newcommand\comments[1]{#1}
\renewcommand\comments[1]{}

% 用于接受从 xelatex/pdflatex 通过参数 -jobname 传入的参数来判定编译何种语言的版本。
% \cnt 的三个参数分别为 en/zh/tw 的内容
\newcommand{\cnt}[3]{{#1}{#2}{#3}}
%\newcommand{\cnt}[3]{#1} % default en
\usepackage{ifthen}
\ifthenelse{\equal{\detokenize{lang-zh}}{\jobname}}{
  \renewcommand{\cnt}[3]{#2}
}{
  \ifthenelse{\equal{\detokenize{lang-tw}}{\jobname}}{
    \renewcommand{\cnt}[3]{#3}
  }{
    % default en
    \renewcommand{\cnt}[3]{#1}
    %\renewcommand{\cnt}[3]{#2}
  }
}

% 根据配置来设置中文环境
\newcommand{\zhenv}[1]{}
\cnt{}{\renewcommand{\zhenv}[1]{#1}}{\renewcommand{\zhenv}[1]{#1}}

\zhenv{

    % 将默认的英文重定义为中文
    \renewcommand{\contentsname}{\cnt{Content}{目录}{目錄}}
    \renewcommand{\listfigurename}{\cnt{List of Figures}{插图目录}{插圖目錄}}
    \renewcommand{\listtablename}{\cnt{List of Tables}{表格目录}{表格目錄}}
    \renewcommand{\indexname}{\cnt{Index}{索引}{索引}}
    \renewcommand{\tablename}{\cnt{Table}{表}{表}}
    \renewcommand{\figurename}{\cnt{Figure}{图}{圖}}
    \renewcommand{\appendixname}{\cnt{Chapter}{附录}{附錄}}

    % article
    \renewcommand{\refname}{\cnt{Bibliography}{参考文献}{參考文獻}}
    \renewcommand{\abstractname}{\cnt{Abstract}{摘要}{摘要}}

    % book
    %\renewcommand{\chaptername}{\cnt{Chapter}{章节}{章節}}
    %\renewcommand{\bibname}{\cnt{Bibliography}{参考}{參考}}

    %\renewcommand{\IEEEkeywordsname}{\cnt{Keywords}{关键词}{關鍵詞}}
}

\zhenv{
\usepackage{ifxetex}
%------------------------------------------------------------------------------------------------------
%U no, it's happy to use chinese now.
%需要注明的是,这种方法是我暂时惟一发现中文XeLaTeX对moderncv原样支持的方式。
%另外几种方式,我得花时间再去试一下,有很多问题,应该只是宏包的冲突。
% 2009-10-3 3:41:57 Saturday
%~ \usepackage{xltxtra,fontspec,xunicode}    %Xelatex支持
%~ \usepackage{zhfont}                       %用zhspacing包支持,之后我再去看 另一种实现是否有问题。
%~ \zhspacing
%\usepackage{xltxtra,xunicode}

\ifxetex % xelatex
    %% chinese setup
    \usepackage[cm-default]{fontspec} % XeLaTex 配合 fontspec 可以非常方便地設置字體。 [cm-default] 選項主要用來解決使用數學環境時數學符號不能正常顯示的問題
    \usepackage{xltxtra,xunicode} % 這行和上行 \usepackage[cm-default]{fontspec} 解決公式不正常的問題.但是打開後有些如 itemize 的點不能顯示。

    \usepackage[
        BoldFont,   % 允許粗體
        SlantFont,  % 允許斜體
        %CJKsetspaces,
        CJKchecksingle
        ]{xeCJK}
    \defaultfontfeatures{Mapping=tex-text} %如果沒有它,會有一些 tex 特殊字符無法正常使用,比如連字符。

    \XeTeXlinebreaklocale "zh" % 重要,使得中文可以正確斷行!
    \XeTeXlinebreakskip = 0pt plus 1pt minus 0.1pt

%----------------------------------------------------------------------------------------
% 以下是设置可能用到的字体,但若无其它设定,本文档所用之默认字体为
% $zhspacing/zhfont.sty文件中的第71行附近,关于zhsffont的定义
% \newfontfamilywithslant\zhsffont{Adobe Heiti Std R}
% 即以下四行配置无实际作用。zhfont中设定优先级更高。
% zhspacing文档的解释是对于XeLaTeX和XeLaTeX对应的宏方案不同。
% 亦不支持\font\myname={"SuSE"}型宏。
    %\setCJKmainfont[BoldFont={WenQuanYi Micro Hei}]{AR PL UMing CN}
    %\setCJKsansfont{AR PL UMing CN} %{Microsoft YaHei}
    %\setCJKmonofont{WenQuanYi Micro Hei Mono}
    \setCJKmainfont[BoldFont={Adobe Heiti Std}, ItalicFont={Adobe Kaiti Std}]{Adobe Song Std}
    \setCJKsansfont{Adobe Ming Std} %{AR PL UMing CN} %{Microsoft YaHei}
    \setCJKmonofont{Adobe Fangsong Std}

    \setmainfont{Times New Roman}
    \setsansfont{DejaVu Sans}
    \setmonofont{FreeMono}%{Latin Modern Mono}
    %\setsansfont{[foo.ttf]} % 直接使用当前目录下的字体文件

%~ \setmainfont{Scala}           %英文字体
%~ \setmonofont{Nimbus Sans L}                     %英文等宽体
%~ \setsansfont[BoldFont={Myriad Pro}]{Myriad Pro}    %不注释了 线性非线性,这边没有限制。
%----------------------------------------------------------------------------------------
% 如果需要,可以通过设置选项 roman/san 更换main字体。
% 如果要指定中文字体,可以另定义字体宏:
%~ \usepackage{fontspec}  
%~ \newfontfamily\ni{"SimHei"}

    \defaultfontfeatures{Mapping=tex-text}
\fi
} % \zhenv

% the algorithm2e package
\makeatletter
\newif\if@restonecol
\makeatother
\let\algorithm\relax
\let\endalgorithm\relax
%%\usepackage[figure,ruled,vlined]{algorithm2e}
\usepackage[ruled,vlined]{algorithm2e}

\usepackage{ifthen}
\usepackage{ifpdf}
\usepackage{ifxetex}
\usepackage{ifluatex}

\ifxetex % xelatex
\else
    %The cmap package is intended to make the PDF files generated by pdflatex "searchable and copyable" in acrobat reader and other compliant PDF viewers.
    \usepackage{cmap}%
\fi

\usepackage{amssymb}
%\usepackage{amsmath,amsfonts,amsthm}

%\usepackage[margin=1.2in,nohead]{geometry}
\usepackage[margin=1.2in]{geometry}

\usepackage{booktabs,longtable} % table in seperate pages.
\usepackage{array}
% table's multirow and multicolumn
\usepackage{multirow}

% ============================================
% Check for PDFLaTeX/LaTeX
% ============================================
\newcommand{\outengine}{xetex}
\newif\ifpdf
\ifx\pdfoutput\undefined
  \pdffalse % we are not running PDFLaTeX
  \ifxetex
    \renewcommand{\outengine}{xetex}
  \else
    \renewcommand{\outengine}{dvipdfmx}
  \fi
\else
  \pdfoutput=1 % we are running PDFLaTeX
  \pdftrue
  \usepackage{thumbpdf}
  \renewcommand{\outengine}{pdftex}
\fi
\usepackage[\outengine,
    bookmarksnumbered, %dvipdfmx
    %% unicode, %% 不能有unicode选项,否则bookmark会是乱码
    colorlinks=true,
    citecolor=red,
    urlcolor=blue,        % \href{...}{...} external (URL)
    filecolor=red,      % \href{...} local file
    linkcolor=black, % \ref{...} and \pageref{...}
    breaklinks,
    pdftitle={\doctitle},
    pdfauthor={\docauthor},
    pdfsubject={\docsubject},
    pdfkeywords={\dockeywords},
    pdfproducer={Latex with hyperref},
    pdfcreator={pdflatex},
    %%pdfadjustspacing=1,
    pdfborder=1,
    pdfpagemode=UseNone,
    pagebackref,
    bookmarksopen=true]{hyperref}

% --------------------------------------------
% Load graphicx package with pdf if needed 
% --------------------------------------------
\ifxetex    % xelatex
    \usepackage{graphicx}
\else
    \ifpdf
        \usepackage[pdftex]{graphicx}
        \pdfcompresslevel=9
    \else
        \usepackage{graphicx} % \usepackage[dvipdfm]{graphicx}
    \fi
\fi
%% \DeclareGraphicsRule{.jpg}{eps}{.bb}{}
\DeclareGraphicsRule{.png}{eps}{.bb}{}
\graphicspath{{./} {figures/}}
\usepackage{flafter} % 防止图形在文字前

\usepackage{color}
\usepackage{courier}
\usepackage{listings} % list the source code
\definecolor{ForestGreen}{rgb}{0.13,0.55,0.13}

\lstset{
    language=C,
    captionpos=b,
    tabsize=2,
    frame=lines,
    basicstyle= \normalfont\ttfamily, % \large\ttfamily, % \small\ttfamily, % \footnotesize\ttfamily, % \scriptsize\ttfamily, % Standardschrift,
    keywordstyle=\color{blue},
    commentstyle=\color{ForestGreen},
    stringstyle=\color{red},
    numbers=left,
    numberstyle=\tiny,
    numbersep=5pt,
    breaklines=true,
    showstringspaces=false,
    emph={label}
}

\definecolor{darkgreen}{cmyk}{0.7, 0, 1, 0.5}
\definecolor{darkblue}{rgb}{0.1, 0.1, 0.5}
\lstdefinelanguage{diff}
{
    keywords={+, -, \ , @@, diff, index, new},
    sensitive=false,
    morecomment=[l][""]{\ },
    morecomment=[l][\color{darkgreen}]{+},
    morecomment=[l][\color{red}]{-},
    morecomment=[l][\color{darkblue}]{@@},
    morecomment=[l][\color{darkblue}]{diff},
    morecomment=[l][\color{darkblue}]{index},
    morecomment=[l][\color{darkblue}]{new},
    morecomment=[l][\color{darkblue}]{similarity},
    morecomment=[l][\color{darkblue}]{rename},
}

\usepackage{chapterbib}
\usepackage[sectionbib,super,square,sort&compress]{natbib}

\title{\doctitle}

\date{} %{Feb. 2013} % <--- leave date empty
\author{
  \docauthor%
    \thanks{
        %School of Computing, Clemson University, Clemson, SC 29633, USA, \protect \url{yfu@clemson.edu}
        (v1.1)}
}

\begin{document}

\maketitle

\section{\cnt{Introduction}{介绍}{介紹}}
\cnt{Here we introduce how to use Garmin Forerunner in Linux system.} % en/zh/tw
    {这里要介绍 Garmin Forerunner 在Linux下的使用。}
    {這裏要介紹 Garmin Forerunner 在Linux下的使用。}

\section{\cnt{Linux Setup}{Linux 系统设置}{Linux 系統設置}}

\cnt{You may type the command line:}
    {先输入命令:}
    {先輸入命令:}
\begin{lstlisting}[language=bash]
udevadm monitor
\end{lstlisting}

\cnt{You may type the command line, if you get:}
    {然后插入 ANT+ key,如果得到如下信息:}
    {然後插入 ANT+ key,如果得到如下信息:}
\begin{lstlisting}[language=bash]
monitor will print the received events for:
UDEV - the event which udev sends out after rule processing
KERNEL - the kernel uevent

KERNEL[10617.492679] add      /devices/pci0000:00/0000:00:1a.0/usb1/1-1/1-1.2 (usb)
KERNEL[10617.493716] add      /devices/pci0000:00/0000:00:1a.0/usb1/1-1/1-1.2/1-1.2:1.0 (usb)
UDEV  [10617.523787] add      /devices/pci0000:00/0000:00:1a.0/usb1/1-1/1-1.2 (usb)
UDEV  [10617.533120] add      /devices/pci0000:00/0000:00:1a.0/usb1/1-1/1-1.2/1-1.2:1.0 (usb)
\end{lstlisting}
\cnt{it means it was not recognized correctly.}
    {表示没有识别正确。}
    {表示沒有識別正確。}

\cnt{You may also type following command to get the USB information after inserting you ANT+ key:}
    {也可以插入 ANT+ key,然后使用如下命令获取信息:}
    {也可以插入 ANT+ key,然後使用如下命令獲取信息:}
\begin{lstlisting}[language=bash]
$ lsusb | grep Dynastream
Bus 001 Device 004: ID 0fcf:1008 Dynastream Innovations, Inc.
\end{lstlisting}

\subsection{\texttt{udev} Rule}
\cnt{If \texttt{usbserial} is a seperated module,}
    {如果 \texttt{usbserial} 是分立的模块,则可以}
    {如果 \texttt{usbserial} 是分立的模組,則可以}
\cnt{you may add a \texttt{udev} rule to the config file \texttt{/etc/udev/rules.d/80-garmin-ant2.rules}:}
    {则可以加入一条 \texttt{udev} 规则到配置文件 \texttt{/etc/udev/rules.d/80-garmin-ant2.rules}:}
    {則可以加入一條 \texttt{udev} 規則到配置文件 \texttt{/etc/udev/rules.d/80-garmin-ant2.rules}:}
\begin{lstlisting}[language=bash]
sudo su -
#not work: echo 'BUS=="usb", SYSFS{idVendor}=="0fcf", SYSFS{idProduct}=="1008", RUN+="/sbin/modprobe usbserial vendor=0x0fcf product=0x1008"' > /etc/udev/rules.d/80-garmin-ant2.rules
#echo 'SUBSYSTEM=="usb", ATTRS{idVendor}=="0fcf", ATTRS{idProduct}=="1008", RUN+="/sbin/modprobe usbserial vendor=0x0fcf product=0x1008"' > /etc/udev/rules.d/80-garmin-ant2.rules
echo 'SUBSYSTEM=="usb", ATTR{idVendor}=="0fcf", ATTR{idProduct}=="1008", MODE="0666", SYMLINK+="ttyANT", ACTION=="add", RUN+="/sbin/modprobe usbserial vendor=0x0fcf product=0x1008"' > /etc/udev/rules.d/80-garmin-ant2.rules
\end{lstlisting}

\cnt{Or you have to type command line at each time you insert your ANT+ key:}
    {或者在每次插入 key 后手工输入:}
    {或者在每次插入 key 後手工輸入:}
\begin{lstlisting}[language=bash]
sudo /sbin/modprobe usbserial vendor=0x0fcf product=0x1008
\end{lstlisting}

\cnt{And the linux should show the messages after you insert your ANT+ key again:}
    {然后重插入 ANT+ key, \texttt{udevadm monitor} 会显示:}
    {然後重插入 ANT+ key, \texttt{udevadm monitor} 會顯示:}
\begin{lstlisting}[language=bash]
KERNEL[3801.729434] add      /module/usbserial (module)
UDEV  [3801.729922] add      /module/usbserial (module)
KERNEL[3801.730226] add      /bus/usb-serial (bus)
UDEV  [3801.730511] add      /bus/usb-serial (bus)
KERNEL[3801.730744] add      /bus/usb/drivers/usbserial (drivers)
KERNEL[3801.730777] add      /bus/usb/drivers/usbserial_generic (drivers)
KERNEL[3801.730803] add      /bus/usb-serial/drivers/generic (drivers)
KERNEL[3801.730913] add      /devices/pci0000:00/0000:00:1a.0/usb1/1-1/1-1.2/1-1.2:1.0/ttyUSB0 (usb-serial)
UDEV  [3801.731572] add      /bus/usb/drivers/usbserial_generic (drivers)
KERNEL[3801.731620] add      /devices/pci0000:00/0000:00:1a.0/usb1/1-1/1-1.2/1-1.2:1.0/ttyUSB0/tty/ttyUSB0 (tty)
UDEV  [3801.732094] add      /bus/usb/drivers/usbserial (drivers)
UDEV  [3801.732279] add      /bus/usb-serial/drivers/generic (drivers)
UDEV  [3801.733116] add      /devices/pci0000:00/0000:00:1a.0/usb1/1-1/1-1.2/1-1.2:1.0/ttyUSB0 (usb-serial)
UDEV  [3801.744466] add      /devices/pci0000:00/0000:00:1a.0/usb1/1-1/1-1.2/1-1.2:1.0/ttyUSB0/tty/ttyUSB0 (tty)
\end{lstlisting}

\cnt{You may check your device files:}
    {然后查看设备文件}
    {然後查看設備文件}
\begin{lstlisting}[language=bash]
$ ls -l /dev/ttyUSB0 
crw-rw---- 1 root dialout 188, 0 May  3 14:02 /dev/ttyUSB0
\end{lstlisting}

\subsection{\texttt{grub2}}

\cnt{If \texttt{usbserial} was compiled in the kernel, as \texttt{Fedora 15} did,}
    {如果 \texttt{usbserial} 模块编译进了内核,如像 \texttt{Fedora 15} 那样,}
    {如果 \texttt{usbserial} 模塊編譯進了內核,如像 \texttt{Fedora 15} 那樣,}
\cnt{then you need to setup \texttt{grub2} \cite{jbrown405linux}.}
    {则需要设置 \texttt{grub2} \cite{jbrown405linux}。}
    {則需要設置 \texttt{grub2} \cite{jbrown405linux}。}
\begin{lstlisting}[language=bash]
# add following line to the GRUB_CMDLINE_LINUX in file /etc/default/grub
# usbserial.vendor=0x0fcf usbserial.product=0x1008
sudo sed -i.bak "s/GRUB_CMDLINE_LINUX_DEFAULT[^ \t]*=[^ \t]*\"/GRUB_CMDLINE_LINUX_DEFAULT=\"usbserial.vendor=0x0fcf usbserial.product=0x1008 /g" /etc/default/grub

# Update the grub config by issuing
grub2-mkconfig > /boot/grub2/grub.cfg

# reboot
shutdown -r now
\end{lstlisting}

\section{\cnt{Transfer Files from Garmin Watch}{从表中获取文件}{從錶中獲取檔案}}
\cnt{You may use Gant\cite{gant405} or \texttt{python-ant-downloader}\cite{braiden405cxsrc} to transfer the files from the watch to your computer.}
    {如果要从 Garmin Forerunner 405 传输文件,可以使用 Gant\cite{gant405} 或者 \texttt{python-ant-downloader}\cite{braiden405cxsrc} 来从表中获取文件。}
    {如果要從 Garmin Forerunner 405 傳輸文件,可以使用 Gant\cite{gant405} 或者 \texttt{python-ant-downloader}\cite{braiden405cxsrc} 來從表中獲取文件。}
\cnt{If the watch is 610, 310, FR60, 910XT etc, you would like to try \texttt{Garmin-Forerunner-610-Extractor}\cite{gant610}.}
    {如果使用 610, 310, FR60, 910XT 等新型号的,则可以使用 \texttt{Garmin-Forerunner-610-Extractor}\cite{gant610}。}
    {如果使用 610, 310, FR60, 910XT 等新型號的,則可以使用 \texttt{Garmin-Forerunner-610-Extractor}\cite{gant610}。}

\cnt{NOTE: I use 405, so {Garmin-Forerunner-610-Extractor} would not work, and {python-ant-downloader} failed getting files from the watch.}
    {注意,我是使用 405 测试的,所以 {Garmin-Forerunner-610-Extractor} 和 {python-ant-downloader} 都没有测试成功。}
    {注意,我是使用 405 測試的,所以 {Garmin-Forerunner-610-Extractor} 和 {python-ant-downloader} 都沒有測試成功。}
\cnt{The only work gant verion is yhfudev's.}
    {而Gant的几个版本中,只有 yhfudev 的 Gant 版本 \cite{gant405} 可以成功获取文件。}
    {而Gant的幾個版本中,只有 yhfudev 的 Gant 版本 \cite{gant405} 可以成功獲取文件。}

\subsection{\cnt{Gant}{使用 Gant}{使用 Gant}}
\comments{
\cnt{You may install the gant for your Debian Linux:}{可以直接在debian下安装:}{可以直接在debian下安裝:}
\begin{lstlisting}[language=bash]
sudo apt-get install -y garmin-ant-downloader
\end{lstlisting}
}

\cnt{If you want to compile and run from source code, you may want to use yhfudev's gant version \cite{gant405}.}
    {如果是选择源代码编译安装,可以选择 yhfudev 的 gant 版本 \cite{gant405}。}
    {如果是選擇源代碼編譯安裝,可以選擇 yhfudev 的 gant 版本 \cite{gant405}。}

\cnt{Download and compile gant}{下载编译:}{下載編譯:}
\begin{lstlisting}[language=bash]
mkdir -p ~/Development/Native/
cd ~/Development/Native/
git clone https://github.com/yhfudev/gant.git gant-git
cd gant-git/
mkdir -p build
cd build
../configure && make
\end{lstlisting}

\cnt{Run}{使用}{運行}
\begin{enumerate}
  \item \cnt{Insert the ANT+ key to the computer's USB slot;}{插入 ANT+ key 到计算机 USB 接口上;}{插入 ANT+ key 到電腦 USB 接口上;}

  \item \cnt{Put Forerunner into pairing mode}{在表的菜单下设置 ANT+ 的对接:}{在表的菜單下設置 ANT+ 的對接:}
\begin{lstlisting}[language=bash]
Menu > Settings > ANT+ > Computer > Pairing > On
\end{lstlisting}

  \item \cnt{make sure that communication is enabled:}{确保打开了通讯:}{確保打開了通訊:}
\begin{lstlisting}[language=bash]
Menu > Settings > ANT+ > Computer > Enabled > Yes
\end{lstlisting}

  \item \cnt{You may need to also set "Force Send". This seems to send all data on the device, whether or not the data's been downloaded from the device before.}{如果需要的话,打开强制发送所有已发和未发数据:}{如果需要的話,打開強制發送所有已發和未發數據:}
\begin{lstlisting}[language=bash]
Menu > Settings > ANT+ > Computer > Force Send > Yes
\end{lstlisting}

  \item \cnt{authorizated by watch}{表认证:}{錶認證:}
\begin{lstlisting}[language=bash]
# Use garmin-ant-downloader in debian
# gant -f nameofyourwatchid -a nameofauthentificationfile
#sudo ./gant -f Forerunner-405 -a auth405
sudo ./gant -f yhfu -a auth405
\end{lstlisting}

  \item \cnt{transfer .tcx files from your watch}{从表中获取 .tcx 文件}{從錶中獲取 .tcx 文件}
\begin{lstlisting}[language=bash]
sudo ./gant -nza auth405 > output
\end{lstlisting}

\cnt{Alternatively you can move the authorization file to your home directory; Gant looks for it there automatically:}
    {可以将文件 auth405 复制到 HOME 目录而不用再次指定:}
    {可以將文件 auth405 複制到 HOME 目錄而不用再次指定:}
\begin{lstlisting}[language=bash]
cp ./auth405 ~/.gant
sudo ./gant -nz > output
\end{lstlisting}

  \item \cnt{Upload all of these \texttt{.tcx} to \href{http://connect.garmin.com/}{Garmin Connect}, or open these files by pytrainer.}
            {上传这些 .tcx 文件到 \href{http://connect.garmin.com/}{Garmin Connect} 网站上;或使用 pytrainer。}
            {上傳這些 .tcx 文件到 \href{http://connect.garmin.com/}{Garmin Connect} 網站上;或使用 pytrainer。}
\end{enumerate}

\subsection{\cnt{python-ant-downloader}{使用 {python-ant-downloader}}{使用 {python-ant-downloader}}}


\begin{lstlisting}[language=bash]
git clone https://github.com/braiden/python-ant-downloader
sudo apt-get install python-pip python-usb libusb-1.0-0
sudo apt-get install python-lxml
sudo apt-get install python python-lxml python-pkg-resources python-poster python-serial

# replace python-usb 0.4 by 1.0
sudo apt-get install python-pip &&
    pyusbdir="$(mktemp -d)" &&
    pushd "$pyusbdir" &&
    git clone https://github.com/walac/pyusb &&
    cd ./pyusb/; git checkout 1.0.0a3 && cd .. &&
    sudo pip install ./pyusb/ &&
    popd &&
    rm "$pyusbdir" -rfv &&
    unset pyusbdir

./ant-downloader.py --help
\end{lstlisting}

\cnt{Setup the watch as we described above, then run}{象前面说明的那样设置手表端,然后运行}{象前面說明的那樣設置手錶端,然後運行}
\begin{lstlisting}[language=bash]
sudo ./ant-downloader.py
\end{lstlisting}



\subsection{\cnt{Garmin-Forerunner-610-Extractor}{使用 {Garmin-Forerunner-610-Extractor}}{使用 {Garmin-Forerunner-610-Extractor}}}
\cnt{Supported devices by Garmin-Forerunner-610-Extractor \cite{gant610}:}{Garmin-Forerunner-610-Extractor \cite{gant610}所支持的设备:}{Garmin-Forerunner-610-Extractor \cite{gant610}所支持的設備:}
\begin{enumerate}
  \item Garmin Forerunner 60
  \item Garmin Forerunner 405CX
  \item Garmin Forerunner 310XT
  \item Garmin Forerunner 610
  \item Garmin Forerunner 910XT
  \item Garmin FR70
  \item Garmin Swim
\end{enumerate}

\cnt{Download and run the application:}{软件使用:}{軟體使用:}
\begin{lstlisting}[language=bash]
git clone https://github.com/Tigge/Garmin-Forerunner-610-Extractor
cd Garmin-Forerunner-610-Extractor
sudo cp resources/ant-usbstick2.rules /etc/udev/rules.d/80-garmin-ant2.rules
echo 'SUBSYSTEM=="usb", ATTR{idVendor}=="0fcf", ATTR{idProduct}=="1008", MODE="0666", SYMLINK+="ttyANT", ACTION=="add", RUN+="/sbin/modprobe usbserial vendor=0x0fcf product=0x1008"' > /etc/udev/rules.d/80-garmin-ant2.rules

sudo ./garmin.py
\end{lstlisting}


\section{\cnt{Applications}{应用程序}{應用程式}}

\subsection{pytrainer}

\cnt{Pytrainer \cite{pytrainer} is a desktop tool for logging and graphing sporting excursions, the file formats include:}{Pytrainer \cite{pytrainer} 是一个日志图形体育锻炼程序,支持的文件格式包括:}{Pytrainer \cite{pytrainer} 是一個日志圖形體育鍛煉程序,支持的文件格式包括:}
\href{http://www.topografix.com/gpx.asp}{GPX},
\href{http://developer.garmin.com/schemas/tcx/v2/}{TCX},
\href{http://www.thisisant.com/resources/fit}{FIT}
.

\cnt{Install in Ubuntu Linux:}{在 Ubuntu 下可以直接安装:}{在 Ubuntu 下可以直接安裝:}
\begin{lstlisting}[language=bash]
sudo apt-get install pytrainer
\end{lstlisting}

\cnt{Or installed from source code:}{或者通过源代码安装:}{或者通過源代碼安裝:}
\begin{lstlisting}[language=bash]
git clone https://github.com/pytrainer/pytrainer.git
cd pytrainer
sudo python setup.py install
pytrainer -i
\end{lstlisting}

\subsection{Garmin Connect}

\section{\cnt{Related Work}{相关工作}{相關工作}}
\cnt{There's two versions of the gant,}{目前网上有两类版本的 gant 源代码,}{目前網上有兩類版本的 gant 源代碼,}
\cnt{The DanAnker's \footnote{DanAnker's gant \url{https://github.com/DanAnkers/garmin-ant-downloader.git}} seems contains the latest version  of the origin author,}
    {DanAnkers 版本 \footnote{DanAnker's gant \url{https://github.com/DanAnkers/garmin-ant-downloader.git}} 这个似乎更接近原作者最新的版本,}
    {DanAnkers 版本 \footnote{DanAnker's gant \url{https://github.com/DanAnkers/garmin-ant-downloader.git}} 這個似乎更接近原作者最新的版本,}
\cnt{because of the latest commits are in this repo, while the Debin's \footnote{Debian's gant \url{git://git.debian.org/git/pkg-running/garmin-ant-downloader.git}} has not.}
    {因为最后的两个提交在 Debian 的版本库中 \footnote{Debian's gant \url{git://git.debian.org/git/pkg-running/garmin-ant-downloader.git}} 没有。}
    {因為最後的兩個提交在 Debian 的版本庫中 \footnote{Debian's gant \url{git://git.debian.org/git/pkg-running/garmin-ant-downloader.git}} 沒有。}
\cnt{Another jamesarbrown's \footnote{jamesarbrown's gant \url{https://github.com/jamesarbrown/Gant.git}} was from one early version of wbell in a web forum.}
    {另外一个 jamesarbrown 的版本是源于某论坛上 wbell 的一个早期版本 \footnote{jamesarbrown's gant \url{https://github.com/jamesarbrown/Gant.git}}。}
    {另外一個 jamesarbrown 的版本是源於某論壇上 wbell 的一個早期版本 \footnote{jamesarbrown's gant \url{https://github.com/jamesarbrown/Gant.git}}。}

\cnt{Unfortunally, the three versions above were not able to get files from the watch}{可惜的是上述三个版本都没测试成功,}{可惜的是上述三個版本都沒測試成功,}
\cnt{Only yhfudev's version can work correctly.}
    {只有 yhfudev 的版本
    %\footnote{yhfudev's gant \url{https://github.com/yhfudev/gant.git}}
    \cite{gant405}
    才可以获取文件。}
    {只有 yhfudev 的版本 \cite{gant405} 才可以獲取文件。}

\cnt{James A R Brown \cite{jbrown405linux} described how to transfer files from Garmin watch, and import the files to pytrainer.}
    {James A R Brown 在 \cite{jbrown405linux} 中介绍如何在 Linux Fedora 下使用 Gant 从Garmin 手表中获取文件,并且将获得的文件导入 pytrainer 中使用。}
    {James A R Brown 在 \cite{jbrown405linux} 中介紹如何在 Linux Fedora 下使用 Gant 從Garmin 手表中獲取文件,並且將獲得的文件導入 pytrainer 中使用。}

\cnt{braiden \cite{braiden305doc} introduced how to use Garmintools \cite{dbailegarmintools} to communicate with Garmin Forerunner 305\cite{garmin305} in Linux, and he also give an enhanced Python tool\cite{braiden405cxsrc};}
    {braiden \cite{braiden305doc} 首先描述了在 Linux 下如何使用 Garmintools \cite{dbailegarmintools} 来和 Garmin Forerunner 305\cite{garmin305} 通讯,同时他提供了增强的 Python 工具 \cite{braiden405cxsrc};}
    {braiden \cite{braiden305doc} 首先描述了在 Linux 下如何使用 Garmintools \cite{dbailegarmintools} 來和 Garmin Forerunner 305\cite{garmin305} 通訊,同時他提供了增強的 Python 工具 \cite{braiden405cxsrc};}
\cnt{Another work \cite{braiden405cxdoc} introduced the protocol between host and Garmin Forerunner 405\cite{garmin405}CX\cite{garmin405cx}, and it also introduced the Python source code\cite{braiden405cxsrc}.}
    {在另外一篇 \cite{braiden405cxdoc} 介绍了 Garmin Forerunner 405\cite{garmin405}CX\cite{garmin405cx} 在 Linux 下的通讯协议以及为此实现的 Python 源代码\cite{braiden405cxsrc}。}
    {在另外一篇 \cite{braiden405cxdoc} 介紹了 Garmin Forerunner 405\cite{garmin405}CX\cite{garmin405cx} 在 Linux 下的通訊協議以及為此實現的 Python 源代碼\cite{braiden405cxsrc}。}

\cnt{Garmin-Forerunner-610-Extractor \cite{gant610} support a newer devices, such as 610.}
    {Garmin-Forerunner-610-Extractor \cite{gant610} 支持一些新的设备如 610 等的文件传输。}
    {Garmin-Forerunner-610-Extractor \cite{gant610} 支持一些新的設備如 610 等的文件傳輸。}

\cnt{Linux Garmin Communicator Plugin\cite{garminwebfirefoxplugin} support uploading files to Gamin's website by using Firefox.}
    {如果需要 Firefox 支持 Garmin 的网上程序数据上传,可以使用  Linux Garmin Communicator Plugin\cite{garminwebfirefoxplugin}}
    {如果需要 Firefox 支持 Garmin 的網上程序數據上傳,可以使用  Linux Garmin Communicator Plugin\cite{garminwebfirefoxplugin}}
\cnt{\footnote{The Plugin's GitHub source code \url{https://github.com/adiesner/GarminPlugin.git}, to compile it, it needs xulrunner-dev, Tinyxml, garmintools, libusb etc.}}
    {\footnote{GitHub 上的源代码 \url{https://github.com/adiesner/GarminPlugin.git},需要 xulrunner-dev, Tinyxml, garmintools, libusb 等支持。}}
    {\footnote{GitHub 上的源代碼 \url{https://github.com/adiesner/GarminPlugin.git},需要 xulrunner-dev, Tinyxml, garmintools, libusb 等支持。}}
。

%\section{\cnt{Ref}{参考}{參考}}

\bibliographystyle{ieeetr}%{unsrt}
\bibliography{ref}

\end{document}
